\documentclass{article}
\usepackage[T2A]{fontenc}
\usepackage[utf8]{inputenc}
\usepackage[russian,english]{babel}
\usepackage{listings}
\usepackage{indentfirst}
\usepackage{graphicx}
\usepackage{array}
\usepackage{wrapfig}
\usepackage{multirow}
\usepackage{tabularx}
\graphicspath{ {./images/} }
\usepackage{geometry}
\geometry{verbose,a4paper,tmargin=2cm, bmargin=2cm, lmargin=2.5cm, rmargin=1.5cm}

\usepackage{xcolor}
\usepackage{hyperref}
 
 % Цвета для гиперссылок
\definecolor{linkcolor}{HTML}{799B03} % цвет ссылок
\definecolor{urlcolor}{HTML}{799B03} % цвет гиперссылок
 
\hypersetup{pdfstartview=FitH,  linkcolor=linkcolor,urlcolor=urlcolor, colorlinks=true}

\definecolor{codegreen}{rgb}{0,0.6,0}
\definecolor{codegray}{rgb}{0.5,0.5,0.5}
\definecolor{codepurple}{rgb}{0.58,0,0.82}
\definecolor{backcolour}{rgb}{0.95,0.95,0.92}

\lstdefinestyle{mystyle}{
    backgroundcolor=\color{backcolour},   
    commentstyle=\color{codegreen},
    keywordstyle=\color{magenta},
    numberstyle=\tiny\color{codegray},
    stringstyle=\color{codepurple},
    basicstyle=\ttfamily\footnotesize,
    breakatwhitespace=false,         
    breaklines=true,                 
    captionpos=b,                    
    keepspaces=true,                 
    numbers=left,                    
    numbersep=5pt,                  
    showspaces=false,                
    showstringspaces=false,
    showtabs=false,                  
    tabsize=2
}

\lstset{
language = C++,
style = mystyle,
extendedchars=\true}



\title{Оценка погрешности алгоритмов численного интегрирования на языке С++ с использованием OpenMP}
\author{Оришин И.C.}
\date{Июнь 2020}


\begin{document}

\maketitle
\newpage
\section{Погрешности формул численного интегрирования}
\subsection{Общая формула погрешностей}
Для оценки погрешности квадратурных и кубатурных формул обычно используется 
следующая формула оценки:
\begin{center}
$$\delta = max(|f^{(n)}_{x_i}(x_1,...,x_s)|)\frac{(b-a)^k}{CN^n} $$
\end{center}

где:
\begin{itemize}
  \item $f(x_1,...,x_s)$ – подинтегральная функция;
  \item s – количество независимых переменных;
  \item n – порядок производной;
  \item b – верхний предел интегрирования;
  \item a - нижний предел интегрирования;
  \item k - степень, в которую возводится разница пределови интегрирования;
  \item С - некоторая константа;
  \item N - количество разбиений.
\end{itemize}
Разным методам соответствуют разные значения параметров.
\begin{table}[h!]
\centering
\begin{tabular}{|l|l|l|l|}
\hline
Метод  & n & k & C \\ \hline
Левых прямоугольников & 1    & 2    & 2    \\ \hline
Правых прямоугольников & 1    & 2    & 2    \\ \hline
Средних прямоугольников & 2    & 3    & 24    \\ \hline
Трапеций & 2    & 3    & 12    \\ \hline
Симпсона & 4    & 5    & 2880    \\ \hline
Ньютона-Котеса & 4    & 5    & 2880    \\ \hline
\end{tabular}
\end{table}
\subsection{Погрешность метода Монте-Карло}
Для метода Монте-Карло погрешность можно оценить следующим образом:
$$\delta = 3\sqrt{\frac{D[X]}{N}} $$
где D[X] - дисперсия случайной величины, а N - количество разбиений.
Метод который используется для генерации чисел, использует
std::uniform\text{_}real\text{_}distribution<>, что в статистике
соответствует равномерному распределению. Дисперсия равномерного распределения
равна:

$$ D[X] = \frac{(b-a)^2}{12} $$

Следовательно формула для оценки погрешности примет вид:

$$\delta = 3\sqrt{\frac{(b-a)^2}{12N}} $$

\subsection{Погрешности одномерных методов}
Подставляя в формулу известные константы из таблицы в общую формулу получим:
для методов левых и правых прямоугольников:
$$\delta = max(|f'_{x}(x)|)\frac{(b-a)^2}{2N} $$
для метода средних прямоугольников:
$$\delta = max(|f''_{x}(x)|)\frac{(b-a)^3}{24N^2} $$
для метода трапеций:
$$\delta = max(|f''_{x}(x)|)\frac{(b-a)^3}{12N^2} $$
для методов Симпсона и Ньютона-Котеса:
$$\delta = max(|f^{(4)}_{x}(x)|)\frac{(b-a)^5}{2880N^4} $$
Для одномерного метода Монте-Карло формула остается неизменной.
\subsection{Погрешности двумерных методов}
Погрешности двумерных методов, ничем не отличаются. Идея в следующем:
\begin{itemize}
  \item посчитать погрешность по переменной x;
  \item посчитать погрешность по переменной y;
  \item сложить результаты.
\end{itemize}
Итого имеем для методов левых и правых прямоугольников:
$$\delta = max(|f'_{x}(x)|)\frac{(b-a)^2}{2N}+max(|f'_{y}(y)|)\frac{(d-c)^2}{2N}  $$
для метода средних прямоугольников:
$$\delta = max(|f''_{x}(x)|)\frac{(b-a)^3}{24N^2}+max(|f''_{y}(y)|)\frac{(d-c)^3}{24N^2} $$
для метода трапеций:
$$\delta = max(|f''_{x}(x)|)\frac{(b-a)^3}{12N^2}+max(|f''_{y}(y)|)\frac{(d-c)^3}{12N^2} $$
для методов Симпсона и Ньютона-Котеса:
$$\delta = max(|f^{(4)}_{x}(x)|)\frac{(b-a)^5}{2880N^4}+max(|f^{(4)}_{y}(y)|)\frac{(b-a)^5}{2880N^4} $$
Для двумерного метода Монте-Карло формула остается неизменной.
\subsection{Программная реализация формул}
Для погрешностей заранее посчитаны максимальные абсолютные значения функций(dfx, d2fx, d4fx, dfy, d2fy, d4fy).
Ознакомться с кодом можно здесь:
\newline
\href{https://github.com/KnsRoo/Parallel-numerical/blob/master/ParaNP/diffs.py}{https://github.com/KnsRoo/Parallel-numerical/blob/master/ParaNP/diffs.py}
\begin{lstlisting}
    //Монте-Карло
    result = 3*sqrt((pow((UPPER_LIMIT-LOWER_LIMIT),2)/12*parts);
    //Одномерные
    //Левых и правых
    result = dfx*(pow(UPPER_LIMIT-LOWER_LIMIT,2))/(2*parts)
    //Средних
    result = d2fx*(pow(UPPER_LIMIT-LOWER_LIMIT,3))/(24*pow(parts,2));
    //Трапеций
    result = d2fx*(pow(UPPER_LIMIT-LOWER_LIMIT,3))/(12*pow(parts,2));
    //Симпсона
    result = d4fx*(pow(UPPER_LIMIT-LOWER_LIMIT,5))/(2880*pow(parts,4));
    //Двумерные
    //Левых и правых
    result = dfx*(pow(UPPER_LIMIT-LOWER_LIMIT,2))/(2*parts)+dfy*(pow(UP_LINE-DOWN_LINE,2))/(2*parts);
    //Средних
    result = d2fx*(pow(UPPER_LIMIT-LOWER_LIMIT,3))/(24*pow(parts,2))+d2fy*(pow(UP_LINE-DOWN_LINE,3))/(24*pow(parts,2));
    //Трапеций
    result = d2fx*(pow(UPPER_LIMIT-LOWER_LIMIT,3))/(12*pow(parts,2))+d2fy*(pow(UP_LINE-DOWN_LINE,3))/(12*pow(parts,2));
    //Симпсона
    result = (d4fx*(pow(UPPER_LIMIT-LOWER_LIMIT,5)))/(2880*pow(parts,4))+(d4fy*(pow(UP_LINE-DOWN_LINE,5)))/(2880*pow(parts,4));
\end{lstlisting}
\end{document}
